\documentclass{article}
\usepackage[utf8]{inputenc}
\usepackage{url}
\usepackage{tabularx}
\usepackage{geometry}
 \geometry{
 a4paper,
 total={170mm,257mm},
 left=22mm,
 top=24mm,
 textwidth=165mm
 }
\usepackage[dvipsnames]{xcolor}
\usepackage{fancyhdr}
\pagestyle{fancy}
\fancyhf{}
\rhead{\textcopyright Nathan Masters 2020}
\lhead{LEGO Colors}

\title{LEGO\textsuperscript{\textregistered} Color Organization: Family Theory}
\author{Nathan Masters}
\date{January 28th, 2020}

%color definitions
%Template: \definecolor{}{HTML}{}
%Legacy Family
\definecolor{Black}{HTML}{000000}
\definecolor{DKgrey}{HTML}{545955}
\definecolor{LTgrey}{HTML}{969696}
\definecolor{White}{HTML}{F4F4F4}
\definecolor{Red}{HTML}{C91A09}
\definecolor{Blue}{HTML}{0055BF}
\definecolor{Yellow}{HTML}{F2CD37}
\definecolor{Green}{HTML}{237841}
\definecolor{Brown}{HTML}{853D07}
\definecolor{Tan}{HTML}{E8CE98}
\definecolor{Orange}{HTML}{FE8A18}

%Transparent Family
\definecolor{TRdark}{HTML}{BFB7B1}
\definecolor{TRclear}{HTML}{E6E3DA}
\definecolor{TRred}{HTML}{CD544B}
\definecolor{TRblue}{HTML}{7BB6E8}
\definecolor{TRLTblue}{HTML}{AEEADD}
\definecolor{UVblue}{HTML}{CFE2F7}
\definecolor{TRyellow}{HTML}{F7F18D}
\definecolor{TRgreen}{HTML}{84B68D}
\definecolor{TRLTgreen}{HTML}{AFD246}
\definecolor{UVgreen}{HTML}{C0FF00}
\definecolor{TRorange}{HTML}{F08F1C}
\definecolor{UVorange}{HTML}{D9856C}
\definecolor{TRpink}{HTML}{E4ADC8}
\definecolor{TRpurple}{HTML}{96709F}

%Natural Family
\definecolor{SDred}{HTML}{CB9797}
\definecolor{SDblue}{HTML}{95A1BA} %Sand Blue
\definecolor{SDgreen}{HTML}{AACFA8}
\definecolor{SDorange}{HTML}{F0A653}
\definecolor{SDpurple}{HTML}{BFAAC4}
\definecolor{DKred}{HTML}{720E0F}
\definecolor{DKblue}{HTML}{0A3463}
\definecolor{DKgreen}{HTML}{16400B}
\definecolor{DKbrown}{HTML}{352100}
\definecolor{DKtan}{HTML}{9E9964}
\definecolor{DKorange}{HTML}{A95500}
\definecolor{DKpurple}{HTML}{641458}
\definecolor{OVgreen}{HTML}{808000}
\definecolor{DKteal}{HTML}{11557A}
 
 %Lucid Family
%{Turquoise}{HTML}{00CCA4} too bright
\definecolor{Teal}{HTML}{008080}
\definecolor{Azure}{HTML}{078BC9}
\definecolor{Indigo}{HTML}{6874CA}
\definecolor{APPgreen}{HTML}{4B9F4A}
\definecolor{LMgreen}{HTML}{A4BD46}
\definecolor{OJ}{HTML}{FFB400}
\definecolor{Violet}{HTML}{5F0747}
\definecolor{DKpink}{HTML}{DB3978}
\definecolor{Lilac}{HTML}{B57FC1}
\definecolor{Magenta}{HTML}{A90C30}
\definecolor{Coral}{HTML}{FF5971}
\definecolor{SKYblue}{HTML}{7DBFDD}

 %Pastel Family
\definecolor{SPblue}{HTML}{C6D3E5} 
\definecolor{MEDblue}{HTML}{9FC3E9}
\definecolor{SPyellow}{HTML}{FFFFB9}
\definecolor{SPgreen}{HTML}{CFF590}
\definecolor{SPorange}{HTML}{F9BA61} %low production volume, excluded for now
\definecolor{Aqua}{HTML}{E1F7FC}
\definecolor{Lavender}{HTML}{D8A2D6}
\definecolor{PARApink}{HTML}{FFDEFD}
\definecolor{Pink}{HTML}{FFC2E1}
\definecolor{Flesh}{HTML}{F6D7B3}
 
 %Novel Family
\definecolor{PLgrey}{HTML}{5B5D6F}
\definecolor{Copper}{HTML}{D67229}
\definecolor{PLsilver}{HTML}{C0C0C0}
\definecolor{DLsilver}{HTML}{AFAFAF}
\definecolor{CRMsilver}{HTML}{D9D9D9}
\definecolor{PLgold}{HTML}{E3BC51}
\definecolor{DLgold}{HTML}{D8AB32}
\definecolor{CRMgold}{HTML}{FFFCB5}
\definecolor{CRMblue}{HTML}{B3B1FF}
\definecolor{Glow}{HTML}{CAF5AF}
 
%New command used in the table with all available colour names
\newcommand{\thiscolor}[1]{\texttt{#1} \hfill \fcolorbox{black}{#1}{\hspace{5mm}}}

%This changes the row separation in the table
\renewcommand{\arraystretch}{1.5}
 
\begin{document}

\maketitle

\section{Family Groups}
In an effort to refine the process of categorizing and sorting a LEGO\textsuperscript{\textregistered} collection, Nathan Masters has devised a general-purpose color grouping system. His approach seeks to provide a dynamic framework for colors both old and new in the form of groups that summarize multiple properties simultaneously.
\newline
\newline
A selection of 70 widely-used colors are organized into six groupings called families:
\begin{itemize}
    \item Legacy
    \item Transparent
    \item Natural
    \item Lucid
    \item Pastel
    \item Novel
\end{itemize}
The family names help distinguish why certain colors are grouped together and are less strict for the integration of new colors. The groups also provide a rough distribution for piece-per-color occurrence, with higher amounts of parts available in the \textit{Legacy} colors and a decreasing amount of parts available for each family moving right; the \textit{Novel} family being so named for relatively few parts produced in those colors. 
\newline
\newline
These six family groups in particular are designed to strike a balance through the extensive color options available, family sizes are closer than simply grouping similar colors. Intensity also plays a role in these groups, helping distinguish the washed-out \textit{Pastel} pieces from the vibrant \textit{Lucid} hues. Colors in the \textit{Natural} family are usually well-suited to projects intended to model real life, yet the group is diverse enough to include the dark "Earth" colors alongside the muted "Sand" tones.
\newline
\newline
The table on the next page lists abbreviated names for each member of the color families. A sample box with the assigned hexadecimal color is included alongside each entry. The hex values used are a mix of official LEGO\textsuperscript{\textregistered} color codes (Source 1) and Nathan's personal representation. 

\clearpage
\section{Colors in Families}
\begin{center}
\begin{tabular}{|| l | l | l | l | l | l ||} 
%left aligned columns with line separator
\hline
Legacy & Transparent & Natural & Lucid & Pastel & Novel \\ %headers
\hline
\thiscolor{Black} & \thiscolor{TRdark} & \thiscolor{SDred} & \thiscolor{Coral} & \thiscolor{SPblue} & \thiscolor{Copper} \\
\thiscolor{DKgrey} & \thiscolor{TRclear} & \thiscolor{SDpurple} & \thiscolor{Violet} & \thiscolor{MEDblue} & \thiscolor{PLgrey} \\
\thiscolor{LTgrey} & \thiscolor{UVblue} & \thiscolor{SDblue} & \thiscolor{OJ} & \thiscolor{Lavender} & \thiscolor{PLsilver} \\
\thiscolor{White} & \thiscolor{TRLTblue} & \thiscolor{SDgreen} & \thiscolor{LMgreen} & \thiscolor{Pink} & \thiscolor{DLsilver} \\
\thiscolor{Red} & \thiscolor{TRblue} & \thiscolor{SDorange} & \thiscolor{APPgreen} & \thiscolor{PARApink} & \thiscolor{CRMsilver} \\
\thiscolor{Yellow} & \thiscolor{TRpurple} & \thiscolor{DKorange} & \thiscolor{Teal} & \thiscolor{Flesh} & \thiscolor{PLgold} \\
\thiscolor{Blue} & \thiscolor{TRpink} & \thiscolor{DKred} & \thiscolor{SKYblue} & \thiscolor{SPyellow} & \thiscolor{DLgold} \\
\thiscolor{Green} & \thiscolor{TRred} & \thiscolor{DKpurple} & \thiscolor{Azure} & \thiscolor{SPgreen} & \thiscolor{CRMgold} \\
\thiscolor{Tan} & \thiscolor{UVorange} & \thiscolor{DKblue} & \thiscolor{Indigo} & \thiscolor{Aqua} & \thiscolor{CRMblue} \\
\thiscolor{Brown} & \thiscolor{TRorange} & \thiscolor{DKgreen} & \thiscolor{Lilac} & & \thiscolor{Glow} \\
\thiscolor{Orange} & \thiscolor{TRyellow} & \thiscolor{DKbrown} & \thiscolor{DKpink} & &  \\
 & \thiscolor{UVgreen} & \thiscolor{DKtan} & \thiscolor{Magenta} & &  \\
 & \thiscolor{TRLTgreen} & \thiscolor{OVgreen} & & &  \\
 & \thiscolor{TRgreen} & \thiscolor{DKteal} & & & \\
 & & & & & \\
 \hline
11 members & 14 members & 14 members & 12 members & 9 members & 10 members \\ %family size
%Reordered colors for aesthetics 1-28-2020
\hline
\end{tabular}
\end{center}

Note that the often-contested Dark Grey and Light Grey members are listed only once and placed within the \textit{Legacy} family. This table follows Nathan's personal preference and sorting practices, but he suggests that the distinction could be adopted by placing the old versions of the greys in \textit{Legacy} and the updated greys in the \textit{Natural} family. Brown has been treated similarly.
\newline
\newline
LEGO\textsuperscript{\textregistered} has produced far more than 70 individual colors through the years; this family breakdown attempts to cover almost all major pieces with colors available in official products released after plastic mixes became fairly consistent. In the interest of simplicity and accessibility to the general LEGO\textsuperscript{\textregistered} community, some large groups of uncommon colors are excluded:
\begin{itemize}
    \item{Fabuland variants}
    \item{Modulux system}
    \item{Loosely unnofficial colors, like very light grey}
    \item{Single-instance colors found in watches, pens, Clickits, etc}
\end{itemize}

\section{Naming Key}

\begin{l}
\begin{tabular}{|| l | l ||} 
%left aligned columns with line separator
\hline
\textit{Abbreviation} & \textit{Full Convention} \\
\hline
\hline
DK & Dark color \\
\hline
LT & Light color \\
\hline
TR & Transparent color \\
\hline
TRLT & Transparent Light color \\
\hline
UV & Ultraviolet color (aka Neon) \\
\hline
OV & Olive (Green only) \\
\hline
SD & Sand color \\
\hline
LM & Lime (Green) \\
\hline
APP & Apple (Green) \\
\hline
SP & Spring color \\
\hline
PARA & Paradisa (Pink) \\
\hline
PL & Pearl color (metallic) \\
\hline
DL & Drum-lacquered (metallic) \\
\hline
CRM & Chrome finish \\
\hline
\end{tabular}
\end{l}
\newline 

\section{Additional Notes}
\begin{l}
\begin{tabular}{|| l | l ||} 
%left aligned columns with line separator
\hline
\textit{Listing} & \textit{Clarification} \\
\hline
\hline
SDorange & Commonly known as Nougat \\
\hline
DKcolor & Sometimes called Earth colors \\
\hline
OJ & "Orange Juice" yellow-orange color \\
\hline
SPblue &  Light Royal Blue, recent \\
\hline
MEDblue & Not as light as other spring colors, older \\
\hline
Glow & Glow in the Dark \\
\hline
\end{tabular}
\end{l}
\newline 

\section{Sources}
\begin{enumerate}
\item{\url{http://www.peeron.com/inv/colors}}
\item{\url{https://www.overleaf.com/learn/latex/Using_colours_in_LaTeX}}
\item{\url{https://texblog.org/2011/04/19/highlight-table-rowscolumns-with-color/}}
\item{\url{https://www.overleaf.com/project/5dd721997bc92700010c35c0}}
\item{\url{https://htmlcolorcodes.com/}}
\end{enumerate}

\end{document}
